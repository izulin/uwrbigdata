\documentclass[12pt]{uebung}


\dozent{Przemysław Uznański}
\vorlesung{Algorithms for Big Data}
\semester{Spring Semester 2022}
%\tutoren{Tutoren name}
 
\usepackage{amsmath}
\usepackage{mathtools}
 
 
%\usepackage[utf8]{inputenc}
\usepackage{amsmath, amssymb,wasysym}
%\usepackage{tikz}
\newtheorem{definition}{Definition}
\newtheorem{theorem}{Theorem}

 
%\usepackage{multirow}

\usepackage[english]{babel}

 \begin{document}

 \startnummer{1}
 
\kopf[0]{06/04/2022}{6}

\newcommand{\bigo}{\mathcal{O}}
\renewcommand{\aufgname}{Exercise}

\begin{definition}[Hadamard matrix]
We define $H_1 = \begin{bmatrix} 1 \end{bmatrix}$ and $H_{2n} = \begin{bmatrix} H_n & H_n \\ H_n & -H_n \end{bmatrix}$. We  write $F = \frac{1}{\sqrt{n}} H_n$, dropping $n$ from the index (and assuming $n$ is a power of two).
\end{definition}

\begin{aufg}
Show that $\|F x\|_2 = \|x\|_2$ for any $x \in \mathbb{R}^n$.
\end{aufg}

\begin{aufg}
Show that $FF = I$.
\end{aufg}

\begin{aufg}
Show algorithm that given $x \in \mathbb{R}^n$ computes $Fx$ in time $\bigo(n \log n)$.
\end{aufg}

\vfill

\begin{definition}[Fourier matrix]
Let $\omega = 1^{1/n} = \cos(2\pi/n)+i \sin(2 \pi/n)$. We define $W$ as a $n\times n$ matrix such that 
$W = \frac{1}{\sqrt{n}} \begin{bmatrix} 1 & 1 & 1 & 1 & \hdots & 1\\ 1 & \omega & \omega^2 & \omega^3 & \hdots & \omega^{n-1}\\ 1 & \omega^2 & \omega^4 & \omega^6 & \hdots & \omega^{2(n-1)}\\ 1 & \omega^3 & \omega^6 & \omega^9 & \hdots & \omega^{3(n-1)}\\ \vdots & \vdots & \vdots & \vdots & \ddots & \vdots\\ 1 & \omega^{n-1} & \omega^{2(n-1)} & \omega^{3(n-1)} &  \hdots & \omega^{(n-1)(n-1)} \end{bmatrix}$.
\end{definition}

\begin{aufg}
Show that $\|W x\|_2 = \|x\|_2$ for any $x \in \mathbb{C}^n$.
\end{aufg}

\begin{aufg}
Show that $W \overline{W} = I$.
\end{aufg}

\begin{aufg}
Show algorithm that given $x \in \mathbb{C}^n$ computes $Wx$ in time $\bigo(n \log n)$.
\end{aufg}

\end{document}
