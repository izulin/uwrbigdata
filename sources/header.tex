\usepackage{latexsym}
\usepackage{amsmath,amssymb,amsthm}
\usepackage{epsfig}
\usepackage[right=0.8in, top=1in, bottom=1.2in, left=0.8in]{geometry}
\usepackage{setspace}
\usepackage[utf8]{inputenc}
\usepackage[colorlinks=true,urlcolor=Blue,citecolor=Blue,linkcolor=BrickRed]{hyperref}
\usepackage[dvipsnames,usenames]{xcolor}
%\spacing{1.06}

\newcommand{\handout}[3]{
  \noindent
  \begin{center}
  \framebox{
    \vbox{\vspace{0.25cm}
      \hbox to 5.78in { {University of Wrocław:\hspace{0.12cm}Algorithms for
          Big Data (Spring'22)} \hfill #2 }
      \vspace{0.48cm}
      \hbox to 5.78in { {\Large \hfill #3  \hfill} }
      \vspace{0.42cm}
      \hbox to 5.78in {Lecturer: \emph{Przemysław Uznański}\hfill}\vspace{0.25cm}
      
    }
  }
  \end{center}
  \vspace*{4mm}
}
\newcommand{\Ppb}{\mathbf{Pr}}
\newcommand{\Es}{\mathbf{E}}
\newcommand{\bigo}{\mathcal{O}}
\newcommand{\Var}{\mathbf{Var}}
\newcommand{\sF}{\mathcal{F}}
\newcommand{\Nat}{\mathbf{N}}
\newcommand{\Real}{\mathbf{R}}

\newcommand{\lecture}[2]{\handout{#1}{#2}{Lecture #1}}

\newtheorem{theorem}{Theorem}
\newtheorem{corollary}[theorem]{Corollary}
\newtheorem{lemma}[theorem]{Lemma}
\newtheorem{observation}[theorem]{Observation}
\newtheorem{example}[theorem]{Example}
\newtheorem{definition}[theorem]{Definition}
\newtheorem{claim}[theorem]{Claim}
\newtheorem{fact}[theorem]{Fact}
\newtheorem{assumption}[theorem]{Assumption}
\newtheorem{remark}[theorem]{Remark}
\newtheorem{property}[theorem]{Property}
