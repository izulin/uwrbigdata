\documentclass[12pt]{uebung}


\dozent{Przemysław Uznański}
\vorlesung{Algorithms for Big Data}
\semester{Spring Semester 2022}
%\tutoren{Tutoren name}
 
\usepackage{amsmath}
\usepackage{mathtools}
 
 
%\usepackage[utf8]{inputenc}
\usepackage{amsmath, amssymb,wasysym}
%\usepackage{tikz}
\newtheorem{definition}{Definition}
\newtheorem{theorem}{Theorem}

 
%\usepackage{multirow}

\usepackage[english]{babel}

 \begin{document}

 \startnummer{1}
 
\kopf[0]{04/05/2022}{9}

\newcommand{\bigo}{\mathcal{O}}
\renewcommand{\aufgname}{Exercise}

Recall Hadamard transform, given by a matrix $H$, such that $H_{i,j} = \frac{1}{\sqrt{n}} \cdot (-1)^{\textrm{bc(i \& j)}}$, where $i \& j$ is bit-wise AND of binary representations, and $\text{bc}(x)$ returns number of $1$'s in binary representation. Remember: $H = H^{-1}$, and assume $n$ is power of two.
\begin{aufg}
Recall the tests for bits of $u$ in algorithm for $k=1$ of Fourier transform:
$$b_{i} = 0 \quad\quad\text{iff}\quad\quad |a_r - a_{r+n/2^{i+1}}| \le |a_r + a_{r+n/2^{i+1}}|$$
where $r$ is randomly picked. Design analogous test for Hadamard transform.
\end{aufg}

\begin{aufg}[2 pts]
Let $(\hat{a}_0,\ldots,\hat{a}_{n-1})$ be a Hadamard transform of $(a_0,\ldots,a_{n-1})$. Let $m \le n$ be power of two as well. Let $(b_0,\ldots,b_{n-1})$ be a sequence such that for any $0 \le i < n/m$, $(b_{im},b_{im+1},\ldots,b_{im+m-1})$ is a Hadamard transform of $(a_{im},a_{im+1},\ldots,a_{im+m-1})$. 

Show that for any $0 \le j < m$, $(b_{j},b_{m+j}, b_{2m+j}, \ldots, b_{n-m+j})$ is a Hadamard transform of\\ $(\hat{a}_{j},\hat{a}_{m+j}, \hat{a}_{2m+j}, \ldots \hat{a}_{n-m+j})$. (Keep in mind those transforms are of smaller dimension.)
\end{aufg}

\begin{aufg}[2 pts]
Using previous exercise, design sparse Hadamard transform algorithm (it's almost 1-1 equivalent to the one from the lecture).
\end{aufg}

\end{document}
