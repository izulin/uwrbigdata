\documentclass[12pt]{uebung}


\dozent{Przemysław Uznański}
\vorlesung{Algorithms for Big Data}
\semester{Spring Semester 2022}
%\tutoren{Tutoren name}
 
\usepackage{amsmath}
\usepackage{mathtools}
 
 
%\usepackage[utf8]{inputenc}
\usepackage{amsmath, amssymb,wasysym}
%\usepackage{tikz}
\newtheorem{definition}{Definition}
\newtheorem{theorem}{Theorem}

 
%\usepackage{multirow}

\usepackage[english]{babel}

 \begin{document}

 \startnummer{1}
 
\kopf[0]{08/06/2022}{14}
\newcommand{\bigo}{\mathcal{O}}
\renewcommand{\aufgname}{Exercise}


Here is a formalization of MPC model (one of many possible, equivalent):

\begin{itemize}
\item Input size $N$, distributed among machines.
\item Machine memory is $S = N^{\alpha}$ for some $0 < \alpha < 1$.
\item Machines are numbered with unique ID's, $1\ ..\ \frac{N}{S}$.
\item After each round machines send messages addressed to other machines. Each machine can send $\bigo(S)$ atomic messages  in total and receive $\bigo(S)$ atomic messages in total.
\end{itemize}

\begin{aufg}
Show an algorithm for maximum computation: input array $x[1\ ..\ N]$. Output: $\max\{x[i]\}$ (on a single machine), in time $\bigo(\frac{1}{\alpha})$.
\end{aufg}

\begin{aufg}
Show an algorithm for broadcasting: as an input one machine has a message $m$ of size $\bigo(S)$. Output: all machines have $m$. Show $\bigo(\frac{1}{\alpha})$ algorithm.
\end{aufg}

\begin{aufg}
Show that broadcasting cannot be done faster: that there is no $o(\frac{1}{\alpha})$ algorithm.
\end{aufg}

\begin{aufg}
Show an algorithm for prefix sums:  input array $x[1\ ..\ N]$. Output: array $y[1\ ..\ N]$ where $y[i] = x[1]+\ldots+x[i]$. Time: $\bigo(\frac{1}{\alpha})$.
\end{aufg}

\begin{aufg}
Show an algorithm for offsets: input array $x[1\ ..\ N]$ and $S$ values $a_1,\ldots,a_S$. Output: values $j_1,\ldots,j_S$ where $j_k$ is the position of $a_k$ in sorted $x[1\ ..\ N]$. Time: $\bigo(\frac{1}{\alpha})$.
\end{aufg}

\begin{aufg}
Show an algorithm for pivot: input array $x[1\ ..\ N]$ and $S$ values $a_1,\ldots,a_{S-1}$. Output: reshuffle $x$ so that some prefix of machines holds all the values from $x$ smaller than $a_1$, then next batch of machines holds all values from $x$ between $a_1$ and $a_2$, etc. Time: $\bigo(\frac{1}{\alpha})$.
\end{aufg}

\begin{aufg}[2 pts.]
Sorting: input array $x[1\ ..\ N]$. Output: $x$ sorted. Time: $\bigo(\frac{1}{\alpha^2})$. Idea:
\begin{itemize}
\item Pick sample of size $S$. 
\item Use it as a pivot. 
\item Show that whp subproblems are of size $\widetilde\bigo(\frac{N}{\sqrt{S}})$.
\item Recurse on subproblems.
\end{itemize}
\end{aufg}
\end{document}
